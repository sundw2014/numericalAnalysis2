\documentclass[UTF8,twocolum,titlepage]{ctexart}
\usepackage{graphicx}
\usepackage{placeins}
\usepackage{float}
\usepackage{enumerate}
\usepackage{geometry}
\usepackage{amsmath,bm}
\usepackage{url}
% \setCJKfamilyfont{deffont}{simkai.ttf}
\geometry{left=3cm,right=3cm,top=2.5cm,bottom=2.5cm}
\title{自然对数的数值计算}
\author{孙大伟 2014010782}
\date{}

\begin{document}
\maketitle
\boldmath
\section*{需求分析}
\paragraph{}
使用数值方法计算$ln\left(x\right)$,要求精度高于$10^{-20}$。
\section*{方案设计}
\paragraph{}
输入$x$,然后选取一种计算方法进行计算。计算方法包括泰勒级数法,数值积分法,反双曲正切法。由于编程语言内建的浮点数类型不足以支持作业要求的精度,所以需要实现任意精度的四则运算。
\paragraph{}
因为输入的$x$范围较大,不便于控制误差,所以在每种方法计算之前都作变换$$t=\frac{x}{e^{ceil\left(ln\left(x\right)\right)}}$$ $$ln\left(x\right)=ln\left(t\right)+e*ceil\left(ln\left(x\right)\right)$$其中$ceil\left(f\right)$表示对$f$上取整,此变换将操作数变换到区间$\left(\frac{1}{e},1 \right]$上。以下对每种方法的介绍都是假设操作数在区间$\left(\frac{1}{e},1 \right]$上。由于$e$的值可以通过其他途径得到,且除法的误差不做考虑,所以此变换的误差可忽略。
\subsection*{泰勒级数法}
对$ln\left(t\right)$做泰勒展开,因为$t\in\left(\frac{1}{e},1 \right]$,所以泰勒级数收敛到$ln\left(t\right)$。
$$ln\left(t\right)=\sum\limits_{n=1}^{\infty}\frac{\left(-1\right)^{n-1}*\left(t-1\right)^{n}}{n}$$
\subsection*{数值积分法}
因为$$ln\left(t\right)=\int_1^t \frac{1}{x}\,dx$$所以可以使用数值积分方法得到$ln\left(t\right)$的值。实验中使用了龙贝格算法进行计算,即
\begin{eqnarray*}
R_n&=&\frac{64}{63}C_{2n}-\frac{1}{63}C_{n}\\
&=&\frac{64*16}{63*65}S_{4n}-\frac{64+16}{63*65}S_{2n}+\frac{1}{63*65}S_{n}
\end{eqnarray*}
其中
$$S_n=\frac{h}{6}\left[f(a)+f(b)+2\sum_{k=1}^{n-1}f(x_{k})+4\sum_{k=0}^{n-1}f(x_{k}+\frac{h}{2})\right]$$
\subsection*{反双曲正切法}
通过查阅资料,发现了比泰勒级数更有效\cite{Logarithm_Wikipedia}的级数方法,基于反双曲正切的幂级数。
\begin{eqnarray*}
ln\left(t\right) &=&2*artanh\left(\frac{t-1}{t+1}\right)\\&=&2*\sum_{n=0}^{\infty}\frac{1}{2n+1}\left(\frac{t-1}{t+1}\right)^{2n+1}
\end{eqnarray*}

\section*{流程图}

\section*{误差估计}
\subsection*{泰勒级数法}
\subsubsection*{方法误差}
假设使用$N$项泰勒级数进行近似,则余项为级数$$R_N=\sum\limits_{n=N+1}^{\infty}\frac{\left(-1\right)^{n-1}*\left(t-1\right)^{n}}{n}$$
当$N$足够大时,此级数是绝对值单调下降的交错级数,即莱布尼茨级数。所以有结论$$\left|R_N\right| \le  \frac{\left(-1\right)^{N-1}*\left(t-1\right)^{N}}{N}$$对于输入$t\in\left(\frac{1}{e},1 \right]$有$$\left|R_N\right| \le \frac{\left(1-\frac{1}{e}\right)^N}{N}$$实验中取$N=100$所以有$$\left|R_N\right| \le 1.2*10^{-22}$$
\subsubsection*{存储误差}

\subsection*{数值积分法}
\subsubsection*{方法误差}
实验中使用的是龙贝格方法$R_N$,此方法的收敛速度为8阶。所以
\begin{eqnarray*}
R\left[f\right] &\le& h^8\\&=&\left(\frac{1-t}{N}\right)^8\\ &\le& \left(\frac{1-\frac{1}{e}}{N}\right)^8
\end{eqnarray*}
实验中取$N=300$,所以$$R \le 3.9*10^{-22}$$
\subsubsection*{存储误差}

\subsection*{反双曲正切法}
\subsubsection*{方法误差}
假设使用$N$项级数近似,则余项为$$\left|R_N\right| =  2*\sum_{n=N}^{\infty}\frac{1}{2n+1}\left(\frac{1-t}{t+1}\right)^{2n+1}$$所以
\begin{eqnarray*}
\left|R_N\right| &\le& 2*\sum_{n=N}^{\infty}\frac{1}{2N+1}\left(\frac{1-t}{t+1}\right)^{2n+1} \\ &=& \frac{2}{2N+1}\frac{\left(\frac{1-t}{t+1}\right)^{2N+1}}{1-\left(\frac{1-t}{t+1}\right)^2}\\&\le& \frac{2}{2N+1}\cdot\frac{\left(\frac{1-\frac{1}{e}}{\frac{1}{e}+1}\right)^{2N+1}}{1-\left(\frac{1-\frac{1}{e}}{\frac{1}{e}+1}\right)^2}\\&=&\frac{2}{2N+1}\cdot\frac{\left(0.462\right)^{2N+1}}{0.786}
\end{eqnarray*}
实验中取$N=30$,所以$$\left|R_N\right| \le 1.5*10^{-22}$$
\subsubsection*{存储误差}

\section*{结果分析}

\newpage
%如果文档类是article之类的, 用\renewcommand\refname{参考文献}
\renewcommand\refname{参考文献}
\bibliographystyle{plain}
\bibliography{ref.bib}
\end{document}
